\documentclass[12pt]{article}
\usepackage{hyperref}  % add this package to support \autoref
\usepackage{amsmath}
\begin{document}
Hello world! This is my first \LaTeX\ document. 
\newpage
\section{Formulas}
A rectangle has side lengths of $(x+1)$ and $(x+3)$.A hard return is going to start a new paragraph.\\
A rectangle has side lengths of $(x+1)$ and $(x+3)$. \textbackslash\textbackslash\ is a soft return and therefore the line is not indented.

The equation $${A(x)=x^2+4x+3}$$ gives the area of the rectangle.\\
\{\} makes sure to keep your equation on a line.

\begin{equation}\label{eq0}
    \alpha^2+\beta^2=\gamma^2
\end{equation}
Famous Gaussian quadrature:
\begin{equation}
    \begin{split}
        S&=1+2+3+\dots+n\\
        S&=n+(n-1)+(n-2)+\dots+1\\
        2S&=(1+n)+(2+(n-1))+(3+(n-2))+\dots+(n+1)\\
        2S&=n(n+1)\\
        S&=\frac{n(n+1)}{2} 
    \end{split}
\end{equation}
Formulas for various situations:
\begin{equation}
    F(x)=
    \begin{cases}
        0&,\text{if $x<-1$}\\
        x+1&,\text{if $x>3$}\\
        1&,\text{otherwise.}
    \end{cases}
\end{equation}

\[
    a^2+b^2=c^2
\]

Reference test \autoref{eq0}

Insertion of pictures:\\
Try to insert vector graphics so that the image will not change in clarity when it is enlarged or reduced.
\end{document} 